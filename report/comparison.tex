\section{Алгоритм сравнения методов и проверки результатов}\label{cha:comparison/sec:alg}

Пусть мы нашли исходный отрезок (прямую). Идея состоит в том, чтобы спроецировать эту прямую на плоскости точек и найти сумму квадратов расстояний от точек до проекций. Сравнение значений данного функционала будет показывать степень отклонения полученной прямой от исходной или степень погрешности метода.\\

Имеем полученную прямую:

\begin{center}
	$\mathit{l}: \; \begin{cases}
		x = x_0 + m \cdot t \\
		y = y_0 + n \cdot t \\
		z = z_0 + p \cdot t
	\end{cases}$, где $t \in \mathbb{R}$ - параметр. 
\end{center}

Проведем через нашу прямую и направления проектирования плоскости:

\begin{center}
	$\pi_1: \; \begin{cases}
		x = x_0 + m \cdot t_1 + r_1^1 \cdot t_2 \\
		y = y_0 + n \cdot t_1 + r_2^1 \cdot t_2 \\
		z = z_0 + p \cdot t_1 + r_3^1 \cdot t_2
	\end{cases}$, где $t_1, t_2 \in \mathbb{R}$ - параметры. 

	\vspace{0.2cm}
	$\pi_2: \; \begin{cases}
		x = x_0 + m \cdot t_1 + r_1^2 \cdot t_2 \\
		y = y_0 + n \cdot t_1 + r_2^2 \cdot t_2 \\
		z = z_0 + p \cdot t_1 + r_3^2 \cdot t_2
	\end{cases}$, где $t_1, t_2 \in \mathbb{R}$ - параметры. 

	\vspace{0.2cm}
	$\pi_3: \; \begin{cases}
		x = x_0 + m \cdot t_1 + r_1^3 \cdot t_2 \\
		y = y_0 + n \cdot t_1 + r_2^3 \cdot t_2 \\
		z = z_0 + p \cdot t_1 + r_3^3 \cdot t_2
	\end{cases}$, где $t_1, t_2 \in \mathbb{R}$ - параметры. 
\end{center}

Аналогично методу 3 \ref{math3:solution}, найдем нормальные уравнения этих плоскостей:

$$\begin{gathered}
	\pi_1: \; A_1 x + B_1 y + C_1 z + D_1 = 0 \\
	\pi_2: \; A_2 x + B_2 y + C_2 z + D_2 = 0 \\
	\pi_3: \; A_3 x + B_3 y + C_3 z + D_3 = 0
\end{gathered}$$
$$\begin{gathered}
	\begin{cases}
		A_1 = \frac{n r_3^1 - p r_2^1}{\sqrt{(n r_3^1 - p r_2^1)^2 + (p r_1^1 - m r_3^1)^2 + (m r_2^1 - n r_1^1)^2}} \\
		B_1 = \frac{p r_1^1 - m r_3^1}{\sqrt{(n r_3^1 - p r_2^1)^2 + (p r_1^1 - m r_3^1)^2 + (m r_2^1 - n r_1^1)^2}} \\
		C_1 = \frac{m r_2^1 - n r_1^1}{\sqrt{(n r_3^1 - p r_2^1)^2 + (p r_1^1 - m r_3^1)^2 + (m r_2^1 - n r_1^1)^2}} \\
		D_1 = \frac{- (n r_3^1 - p r_2^1) x_0 - (p r_1^1 - m r_3^1) y_0 - (m r_2^1 - n r_1^1) z_0}{\sqrt{(n r_3^1 - p r_2^1)^2 + (p r_1^1 - m r_3^1)^2 + (m r_2^1 - n r_1^1)^2}}
	\end{cases} \\
	\begin{cases}
		A_2 = \frac{n r_3^2 - p r_2^2}{\sqrt{(n r_3^2 - p r_2^2)^2 + (p r_1^2 - m r_3^2)^2 + (m r_2^2 - n r_1^2)^2}} \\
		B_2 = \frac{p r_1^2 - m r_3^2}{\sqrt{(n r_3^2 - p r_2^2)^2 + (p r_1^2 - m r_3^2)^2 + (m r_2^2 - n r_1^2)^2}} \\
		C_2 = \frac{m r_2^2 - n r_1^2}{\sqrt{(n r_3^2 - p r_2^2)^2 + (p r_1^2 - m r_3^2)^2 + (m r_2^2 - n r_1^2)^2}} \\
		D_2 = \frac{- (n r_3^2 - p r_2^2) x_0 - (p r_1^2 - m r_3^2) y_0 - (m r_2^2 - n r_1^2) z_0}{\sqrt{(n r_3^2 - p r_2^2)^2 + (p r_1^2 - m r_3^2)^2 + (m r_2^2 - n r_1^2)^2}}
	\end{cases} \\
	\begin{cases}
		A_3 = \frac{n r_3^3 - p r_2^3}{\sqrt{(n r_3^3 - p r_2^3)^2 + (p r_1^3 - m r_3^3)^2 + (m r_2^3 - n r_1^3)^2}} \\
		B_3 = \frac{p r_1^3 - m r_3^3}{\sqrt{(n r_3^3 - p r_2^3)^2 + (p r_1^3 - m r_3^3)^2 + (m r_2^3 - n r_1^3)^2}} \\
		C_3 = \frac{m r_2^3 - n r_1^3}{\sqrt{(n r_3^3 - p r_2^3)^2 + (p r_1^3 - m r_3^3)^2 + (m r_2^3 - n r_1^3)^2}} \\
		D_3 = \frac{- (n r_3^3 - p r_2^3) x_0 - (p r_1^3 - m r_3^3) y_0 - (m r_2^3 - n r_1^3) z_0}{\sqrt{(n r_3^3 - p r_2^3)^2 + (p r_1^3 - m r_3^3)^2 + (m r_2^3 - n r_1^3)^2}}
	\end{cases}
\end{gathered}$$

Найдем плоскости, которым принадлежат множества точек:
$$\begin{gathered}
	w_1: \; r_1^1 x + r_2^1 y + r_3^1 z - (r_1^1 x_1^1 + r_2^1 y_1^1 + r_3^1 z_1^1) = 0 \\
	w_2: \; r_1^2 x + r_2^2 y + r_3^2 z - (r_1^2 x_1^2 + r_2^2 y_1^2 + r_3^2 z_1^2) = 0 \\
	w_3: \; r_1^3 x + r_2^3 y + r_3^3 z - (r_1^3 x_1^3 + r_2^3 y_1^3 + r_3^3 z_1^3) = 0
\end{gathered}$$

В пересечении плоскостей $\pi_1$ и $w_1$, $\pi_2$ и $w_2$, $\pi_3$ и $w_3$ получаем проекции исходного отрезка $l_1, l_2, l_3$ на плоскости множеств $w_1, w_2, w_3$. Тогда расстояния от точек множества $P_i$ до проекции $l_i$ на соответствующую плоскость $w_i$ равно расстоянию от этих точек до плоскости $\pi_i$, проходящую через искомый отрезок $l$.

$$\rho (p_i^j, l_j) = |A_j x_i^j + B_j y_i^j + C_j z_i^j + D_j|$$
где $i$ - номер соответствующей точки множества,

$j$ - номер соответствующей плоскости $\pi_j$.\\
Тогда функционал суммы квадратов расстояний от точек множеств до проекций исходного отрезка равен:
$$\Lambda = \underset{j=1}{\overset{3}{\sum}}\underset{i=1}{\overset{n_1, n_2, n_3}{\sum}}(A_j x_i^j + B_j y_i^j + C_j z_i^j + D_j)^2$$

Итоговый фунционал получается из полученного выше делением на количество исходных множеств точек и на среднее число точек в каждом множетсве.






































